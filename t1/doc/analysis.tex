\section{Theoretical Analysis}
\label{sec:analysis}

In this section, the circuit shown in Figure~\ref{fig:circuit} is analysed
theoretically, first we aproach the circuit using the mesh analysis, and later we analyse the circuit using the nodal analysis.

\subsection{Mesh Analysis}

As seen during theoretical lessons, we can use a mesh analysis to analyse the circuit.
This method is built upon Kirchhoff's Voltage Law that states:

In a mesh, the sum of all voltages equals 0.

\begin{equation}
  \sum_{i=0}^{n} V_i = 0
  \label{eq:kvl}
\end{equation}

The method consists of identifying every mesh, labeling its current and choosing the currents direction.
Then the KVL equations are written for each mesh, and we can solve the system of equations, solving consequently the circuit.

As seen in in Figure~\ref{fig:mesh}, there are four meshes, each with currents $Y_a$, $Y_b$, $Y_c$, $Y_d$, 
we will label each mesh as A, B, C, D respectively.

%image of meshes
\begin{figure}[H] \centering
  \includegraphics[width=0.6\linewidth]{mesh-currents.pdf}
  \caption{Current choosen for each mesh}
  \label{fig:mesh}
\end{figure}

The equations for each mesh are therefore:

% Mesh Equations
\begin{equation*}
  \mathbf{E_1} : y_1 + R_4\cdot(y_1 - y_3) + R_3 \cdot(y_1 - y_2) + y_1 \cdot R_1 = 0
  \label{eq:kvlA}
\end{equation*}
\begin{equation*}
  \mathbf{E_2} : y_2 \cdot (1-R_3 \cdot K_b) + K_b \cdot R_3 \cdot y_1 = 0
  \label{eq:kvlB}
\end{equation*}
\begin{equation*}
  \mathbf{E_3} : y_3 \cdot R_6 + y_3 \cdot R_7 - V_c + (y_3 - y_1) \cdot R_4 = 0
  \label{eq:kvlC}
\end{equation*}
\begin{equation*}
  \mathbf{E_4} : y_4 = I_d
  \label{eq:kvlD}
\end{equation*}
In determining equation*s $\mathbf{E_1}$ to $\mathbf{E_4}$, we have used the following relations:
\begin{equation*}
  I_b=K_b \times V_b
  \label{eq:extra1}
\end{equation*}
\begin{equation*}
  V_b= R_3 \times (y_3-y_1)
  \label{eq:extra1}
\end{equation*}
\begin{equation*}
  V_c=K_c \times I_c
  \label{eq:extra2}
\end{equation*}
\begin{equation*}
    I_c = y_3
\end{equation*}


In matrix form, the system looks like the following:
\newline
\newline

\center{$\begin{bmatrix}
R_1 + R_2 + R_3  &  -R_3                 &  -R_4        &  0 \\
K_{b} R_3        &  1 - R_{3} K_{b}      &  0           &  0 \\
-R_4             & 0                     &  R_{6} + R_7 + R_4 - K_c            &  0 \\
0 & 0 & 0 & 1
\end{bmatrix}$
$\begin{bmatrix}
y_1 \\
y_2 \\
y_3 \\
y_4
\end{bmatrix}$
$=$
$\begin{bmatrix}
  0 \\
  0 \\
  0 \\
  0
\end{bmatrix}$}
\newline
\newline
\newline
\newline

Solving this system of equations, using the following values generated with the t1\_datagen.py script using the number 93156:


\begin{table}[H]
  \centering
  \begin{tabular}{ c c }
    R1 & 1.03919759193 \\ 
    \hline
    R2 & 2.06836523173 \\ 
    \hline
    R3 & 3.03375774261 \\ 
    \hline
    R4 & 4.12779067183 \\ 
    \hline
    R5 & 3.11985677803 \\ 
    \hline
    R6 & 2.04513887844 \\ 
    \hline
    R7 & 1.04289965713 \\ 
    \hline
    Va & 5.00439410964 \\ 
    \hline
    Id & 1.04536428769 \\ 
    \hline
    Kb & 7.25705461539 \\ 
    \hline
    Kc & 8.23640363075 \\ 
  \end{tabular}
  \caption{Data generated using number 93156}
  \label{tab:data}
\end{table}




We reach the following results, using octave:

 % Table with results created with octave
    \input{../mat/mesh_octave.tex}


\subsection{Nodal Analysis}

\begin{figure}[H] \centering
  \includegraphics[width=0.6\linewidth]{nodal.pdf}
  \caption{Nodes used in Nodal Analysis}
  \label{fig:mesh}
\end{figure}

Reaching the following results with octave:
 % Table with results created with octave
 \input{../mat/nodal_octave.tex}
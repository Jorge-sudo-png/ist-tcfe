\subsection{Step 1: Stationary analysis}    
For $t < 0$ we have that $v_s = V_s = C^{te}$, and also $i_c = \frac{d}{dt}u_c = 0 $ (the capacitor acts as an open circuit), therefore the circuit is equivalent to the one shown in figure (\ref{fig: step1}).

\begin{figure}[H] \centering
  \includegraphics[width=0.6\linewidth]{circuitStep1.pdf}
  \caption{Circuit used to determine all the branch variables for $t < 0$.}
  \label{fig: step1}
\end{figure}    	
Using nodal analysis we get the following system of equations:
    	
$\begin{bmatrix}
1  &  0 &  0 &  0 & 0 & 0 & 0 \\
C_1 & -C_1-C_2-C_3 & C_2 & C_3 & 0 & 0 & 0\\
0 & C_2+K_b & -C_2 & -K_b & 0 & 0 & 0\\
0 & -K_b & 0 & C_5+K_b & -C_5 & 0 & 0\\
0 & 0 & 0 & 0 & 0 & -C_6-C_7 & C_7\\
0 & 0 & 0 & 1 & 0 & K_d C_6 & -1\\
0 & C_3 & 0 & -C_3-C_4-C_5 & C_5 & C_7 & -C_7\\ 
\end{bmatrix}
\begin{bmatrix}
e_1\\
e_2\\
e_3\\
e_5\\
e_6\\
e_8\\
e_7\\
\end{bmatrix}
=
\begin{bmatrix}
V_s\\
0\\
0\\
0\\
0\\
0\\
0\\
\end{bmatrix}$



\subsection{Step 2: Determining the equivalent resistance.}
To get the equation for the capacitor voltage, $v_c$, we need to determine the time constant $\tau = R_{eq}C$, where $C$ is the capacitance and $R_{eq}$ is the resistance measured from the terminals of the capacitor, $v_6, v_8$, however due to the circuit geometry and the presence of dependent sources we can't simplify the circuit directly. To calculate $R_{eq}$, we cancel out the internal sources in the circuit ($V_s$), and apply a known voltage, $V_x$, in the terminals $v_6,\, v_8$, then we simply calculate the ratio between  the voltage from the source and the current that passes through it, $I_x$. To calculate $I_x$ we, again, perform the nodal analysis, but on the modified circuit in figure (\ref{fig: step 2}). Note that $V_x = v_6-v_8$, where $v_6$ and $v_8$ were the nodes voltages obtained in step 1. 

\begin{figure}[H] \centering
  \includegraphics[width=0.6\linewidth]{circuitStep2.pdf}
  \caption{Circuit configuration for measuring the equivalent resistance, $R_{eq}$, from terminals $6$ and $8$.}
  \label{fig: step 2}
\end{figure}

Using nodal analysis we get the following system of equations:
    	
$\begin{bmatrix}
0  &  0 &  0 &  0 & 1 & 0\\
C_6 K_d & 0 & 0 & -1 & 0 & -C_6 K_d\\
-C_1-C_4-C_6 & C_1 & 0 & C_4 & 0 & C_6\\
0 & C_2+ K_b & -C_2 & -K_b & 0 & 0\\
C_1 & -C_1-C_2-C_3 & C2 & C3 & 0 & 0\\
C_6 & 0 & 0 & 0 & 0 & -C_6-C_7\\
\end{bmatrix}
\begin{bmatrix}
e_1\\
e_2\\
e_3\\
e_5\\
e_6\\
e_7\\
\end{bmatrix}
=
\begin{bmatrix}
V_x\\
0\\
0\\
0\\
0\\
0\\
\end{bmatrix}$

The current $I_x$ is then given by the equation:

\begin{equation}
    I_x = C_5(e_5-e_4) 
\end{equation}

\subsection{Step 3: Natural solution}
The natural solution for a capacitor is given by the following equation:

\begin{equation}
    v_{c_n}(t) = V e^{-\frac{1}{RC}t}
\end{equation}
where $V$ is a constant and $R$ is the resistance measured from the terminals of the capacitor.
To determine $V$, we note that the voltage of a capacitor must always be continuous (or else there would be an infinite power in the capacitor in the instance of the discontinuity), therefore if $v_{6_n}(0^-) = v_6_n(0^+)$ and  $v_{6_n}(0^-) = V_x$, then $V = V_x$. 
Therefore the natural solution for the capacitor is given by the equation (\ref{}).
\begin{equation}
    v_{6_n}(t) = V_x e^{-\frac{1}{RC}t}, 
\end{equation}
for $t \geq 0$.


\subsection{Frequency response}
To determine the frequency response of the circuit we have to simplify it, this can be achieved by determining the Thevening equivalent circuit from the terminals $v_6$ and $v_8$, as shown in figure (\ref{Cicuito6.pdf}).

\begin{figure}[H] \centering
  \includegraphics[width=0.6\linewidth]{circuitStep6.pdf}
  \caption{Determining the transfer function, $\bar{T}(w)$, by resorting to circuit simplification, using the Thevenin equivalent( seen from the terminals of the capacitor).}
  \label{fig:mesh}
\end{figure}



The equivalent impedance $Z_{Th}$ has already been determined, its value is equal to the equivalent resistance $R_{eq}$, viewed from the terminals 6 and 8. To determine the value for $U_{Th}$ we use the same system as obtained in step 1 of the theoretical analysis, but with $v_s$, set as a variable instead of a constant. Solving the the system, we get the voltages $v_6$ and $v_8$ as a function of $v_s$, and as expected, the voltage drop $v_6 - v_8$ is proportional to $v_s$.
\\ With the circuit simplified (see figure (\ref{CircuitResp.pdf})), we can relate the source phasor $\bar{U_s}$ with output phasor $\bar{U_c}$, with the folowing equations:

\begin{equation}
    \bar{I_c} = \frac{\alpha \bar{U_s}}{R_{Th}+ \frac{1}{jwC}} \iff \bar{U_c} = \frac{\frac{1}{jwC}}{R_{Th} +  \frac{1}{jwC}} \alpha \bar{U_s}, 
\end{equation}
where $\alpha \bar{U_s} = \bar{U}_{Th}$.
Therefore we get:

\begin{equation}\label{eq: freqResp}
\bar{T}(w) = \frac{\bar{U}_s}{\bar{U}_c} = \frac{\alpha}{j \tau \omega + 1} \end{equation}
, where $\tau = R_{Th}C$.
